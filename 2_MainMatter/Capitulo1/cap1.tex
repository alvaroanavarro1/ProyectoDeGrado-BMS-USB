% ***************************************************************************************
% ************************************* CAPÍTULO I **************************************
% ***************************************************************************************
\chapter{MARCO TEÓRICO}
\thispagestyle{empty}

\abovedisplayskip=0pt
\belowdisplayskip=10pt
\abovedisplayshortskip=0pt
\belowdisplayshortskip=10pt

En este capítulo se exponen los fundamentos teóricos de los diferentes temas relacionados con el trabajo de diseño realizado. 

\section{Sistemas de control}
Un sistema de control está constituido por una cantidad de dispositivos mecánicos, eléctricos, electrónicos, que se organizan por diferentes niveles, pero que en conjunto utilizan un protocolo de comunicación determinado. Estos a su vez proporcionan la capacidad de gestionar y monitorizar diferentes procesos de acuerdo a su estructura utilizada. Históricamente, los sistemas de control se basan en dos capas, la capa de Control Digital Directo (\textit{DDC}) \textit{Direct Digital Control} y la capa de monitorización (\textit{SCADA}) \textit{Supervisory control and data acquisition}(referencia). Con los desarrollos tecnológicos y gracias a la interconexión con sistemas informatizados, se puede puede usar el término de BMS en gestión y supervisión de edificaciones, para definir indistintamente el sistema de control en su totalidad. Para empezar, se puede definir los diferentes niveles que componen un sistema BMS.

\begin{itemize}
    \item \textbf{Nivel de elementos de campo:} 
    este nivel se encarga de actuar o extraer información sobre los dispositivos utilizados en campo. Está compuesto por actuadores, sensores, módulos de E/S del sistema. Los módulos E/S se encargan de recoger las señales de los diferentes sensores y actuadores para incluirlas dentro del bus de comunicaciones del BMS.
    \item \textbf{Nivel de control:}
     está compuesto por la primera serie de controladores parametrizables y PLC. Estos se encargan de la recolección y procesado de datos, permiten también un preprocesamiento y lógicas locales autónomas de las instalaciones. En este nivel es importante garantizar el funcionamiento del sistema ante falla de comunicaciones con controladores de los niveles superiores. Por lo tanto si un controlador maestro falla o existen problemas en el bus de comunicación, el sistema puede seguir funcionando.
     \item \textbf{Nivel gestión:}
     este se compone por PC, servidores o PLC's. A este nivel los dispositivos se encargan de gestionar los distintos controladores utilizados en el nivel de control. Esta etapa es fundamental en términos de monitorización, ya que se recogen los datos de los niveles inferiores y con estos se generan los diferentes medios de presentación de información al usuario. También se le ofrece al usuario una interfaz interactiva con la cual el usuario pueda actuar con simplicidad según los datos visualizados.
     \item \textbf{Nivel usuario:}
     este nivel está compuesto por el conjunto de clientes que tienen acceso al nivel de control, pueden ser PC, móviles o clientes remotos. Los clientes cuentan con una interfaz especializada para conocer el estado total del sistema BMS, se puede configurar también esta interfaz según el tipo de usuario para permitir mayor o menor control sobre el sistema. 
     
    
\end{itemize}

\section{Elementos de un sistema BMS}
\subsection{Elementos de Campo}
\subsection{Elementos de Control}
\subsection{Elementos de Gestión}
\subsection{Elementos de Usuarios}
\section{Elaboración de documentos}
\section{Protocolo BACnet}


