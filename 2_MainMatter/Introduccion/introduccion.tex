% ***************************************************************************************
% ************************************ INTRODUCCIÓN *************************************
% ***************************************************************************************
\chapter*{INTRODUCCIÓN}
\addcontentsline{toc}{chapter}{INTRODUCCIÓN}
\thispagestyle{empty}
Hoy día en el sector de la construcción existe una creciente necesidad, por parte de propietarios e inversionistas, de supervisar, controlar y mejorar las disciplinas de acondicionamiento lumínico y térmico, envueltas en el uso cotidiano de un edificio habitable; impulsando así, desarrollos tecnológicos con el fin de satisfacer dicha necesidad del mercado. 

Una de las alternativas es la instalación de un Sistema de Gestión de Edificaciones o \textit{Building Management System}, llamado \textit{BMS} por sus siglas en inglés. El cual mediante un arreglo de distintos tipos de controladores, permite reportar y detectar fácilmente alteraciones o fallas en algún equipo de acondicionamiento; además de ajustar el funcionamiento adecuadamente según tráfico y horario, lo que permite el ahorro de energía y de recursos necesarios para la operación de las edificaciones.

La empresa Grupo123 C.A, con experiencia en soluciones electromecánicas, se ha fijado en esta oportunidad de mercado. Por ello el departamento de BMS y controles, es el encargado de diseñar la arquitectura correcta para garantizar el control y supervisión eficiente entre los diferentes equipos utilizados. El desarrollo de los diferentes tipos de arquitectura, están basados en los controladores de la línea Facility Explorer de Johnson Controls. Dichos controladores, mediante las facilidades de entradas y salidas analógicas/digitales permiten el control oportuno de los diferentes sub-sistemas; por otro lado, mediante su protocolo de comunicación Bacnet, permite la interconexión entre los diferentes controladores para  centralizar la información de los diferentes equipos.

El objetivo de este proyecto de grado es el de diseñar e implementar una arquitectura capaz de mejorar el desempeño de control en tres diferentes sub-sistemas como lo son: sistema de iluminación, sistema de aire acondicionado y sistema de bombeo primario variable. Dichas técnicas pueden ser replicadas en diferentes instalaciones de plantas de agua helada, eléctricas y de aires acondicionados respectivamente. Se busca también expandir y facilitar las funcionalidades del sistema, para usuarios sin conocimientos previos en materia de controladores. Para lograrlo, se presentarán las variables de operación al usuario mediante una interfaz gráfica, ya que esta representa una manera fácil y amigable de detección de fallas; permitiendo modificar las variables en los diferentes sistemas controlados.\newline

\textbf{Objetivo General:}

\begin{itemize}
    \item Diseño e implementación de un sistema de gestión de edificaciones (BMS) para el control y supervisión de equipos electromecánicos.
\end{itemize}

\textbf{Objetivos Específicos:}

\begin{itemize}
    \item Documentar el estado y lógicas de control de instalaciones previas para sistemas similares a desarrollar..
    \item Automatizar, supervisar y centralizar eficientemente los diferentes sistemas deseados.
    \item Mejorar lógicas de control para reducir el consumo energético. 
    \item Determinar las posibles fallas que pueden afectar los diferentes sistemas de control implementados.
    \item Desarrollar una interfaz con el usuario que brinde información relevante para su revisión.
    \item Definir un procedimiento estándar para el manejo de los diferentes sistemas de control.
    \item Desarrollar lógicas de control para alcanzar eficientemente un sistema de bombeo primario variable. 
\end{itemize}
